% resumo em português

\setlength{\absparsep}{18pt} % ajusta o espaçamento dos parágrafos do resumo
\begin{resumo}

Este estudo apresenta a importância da comunicação e transparência de bancos centrais para a condução da política monetária. Dada esta relevância, foram empregadas técnicas de mineração textual e análise de sentimentos nas atas do Comitê de Política Monetária para criação de um índice, contemplando o período de 2006 até 2022. O índice foi então comparado com variáveis macroeconômicas, e após isso foram efetuados testes estatísticos, para contribuir com a robustez do  trabalho. Como resultados, aplicando Vetores Autorregressivos (VAR), através de Funções de Impulso-Resposta (IRF), pode-se dizer que dado um choque positivo no índice, a diferença da taxa básica de juros real é influenciada negativamente, enquanto que para a atividade econômica e para a produção industrial, percebe-se que a resposta é positiva com impactos descrescentes, onde todas se estabilizam em torno de zero no horizonte analisado, uma vez que os choques não possuem efeitos permanentes em séries estacionárias. 


 \textbf{Palavras-chave}: Mineração Textual. Análise de Sentimentos. Atas do Copom. Vetores Autorregressivos (VAR). Função Impulso-Resposta (IRF).
\end{resumo}

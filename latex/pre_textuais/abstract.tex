% resumo em inglês
\begin{resumo}[Abstract]
 \begin{otherlanguage*}{english}

This research presents the importance around central banking communication and transparency to the conduction of monetary policy. Given this relevance, text mining and sentiment analysis techniques were applied to the Copom minutes in search of an index creation, covering the period from 2006 to 2022. The index was then compared with macroeconomic variables, and after that, statistical tests were performed to contribute to the robustness of the work. As a result, applying Autoregressive Vectors (VAR), through Impulse-Response Functions (IRF), it can be said that given a positive shock on the index, the differenced real basic interest rate is negatively influenced, while for economic activity and for industrial production, it can be seen that the answer is positive with descrescent impacts, where they all stabilize around zero in the analyzed horizon, since the shocks do not have permanent effects in stationary series.

\textbf{Keywords}: Text Mining. Sentiment Analysis. Copom Minutes. Vector Autoregressive (VAR). Impulse Response Function (IRF).
 \end{otherlanguage*}
\end{resumo}


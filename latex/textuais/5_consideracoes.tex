% considerações finais

\chapter{Considerações Finais}

Este estudo buscou elucidar sobre a importância da comunicação e transparência de um Banco Central, isto é, toda informação disponibilizada para os agentes externos, que envolve a condução da política monetária, em relação à atividade econômica e em relação à sinais de trajetórias futuras. Foi apontado, que os níveis atuais de comunicação e transparência foi um trajeto percorrido ao longo dos anos, assim como um movimento internacional.

Dessa forma, como ferramenta de comunicação, o BCB se utiliza das atas do Copom, além da abordagem de transparência adotada por meio do RMI. Com isso, quando se fala da relação entre a discrição e regra, de forma rara um Banco Central seguirá rigorosamente as regras, uma vez que tem-se o chamado viés inflacionário, isto é, o impulso de se diminuir o desemprego ou aumentar o produto. Nesse sentido, quando o RMI não possui credibilidade, a tendência é resultar na discrição, aumentando a oferta de moeda e por consequência gerando uma taxa de inflação maior.

Com isso em mente, apresentaram-se estudos que aplicaram na economia, técnicas de mineração textual, que é um processo para organizar e transformar dados textuais disponíveis em documentos e grandes bases de dados, para extrair significados e padrões. Como as atas do Copom são disponibilizadas no site do BCB, o objetivo deste trabalho foi aplicar essas técnicas para extrair as todas as atas lá presentes, filtrar para o período de 2006 até 2022 e, por meio de análise de sentimentos, criar um índice para os comunicados.

A partir da elaboração do índice, comparou-se ele com variáveis macroeconômicas de interesse, e com os devidos testes econométricos feitos para asseguração da consistência dos resultados, aplicou-se modelagem de vetores autorregressivos, podendo assim inferir ou não causalidade de Granger das variáveis, como também a elaboração de funções de impulso-resposta.

Os critérios de informação para escolha do número de defasagens do modelo apontou para uma e duas, entretanto, como o VAR($1$) falhou no teste de autocorrelação serial dos resíduos, trabalhou-se somente com o VAR($2$). Dessa maneira, foi possível sair de uma simples correlação para uma análise de causalidade, onde obteve-se que o índice Granger-causa a produção industrial e as variáveis de taxa de juros real e produção industrial Granger-causam o índice. Analisando as IRF, para os juros reais, tem-se uma resposta negativa, sendo os choques significativos para o 4º e 5º período; para a atividade econômica, os impactos são sentidos no 2º mês, perdurando significativamente até o 5º período; e por fim, a resposta da produção industrial é afetada positivamente, com um pico de impacto no 5º período.

Enfim, reitera-se que este trabalho procurou contribuir com a aplicação da intersecção de técnicas de mineração textual, análise de sentimentos e macroeconomia, além de considerar técnicas estatísticas-econométricas para contribuir com a robustez das resultantes, com o VAR cumprindo o proposto no trabalho. Como trabalhos futuros, pode-se sugerir modelagens preditivas, levando em consideração que o VAR é um ótimo modelo para tal, assim como a construção de novos índices de sentimentos, com diferentes fontes de mineração textual e/ou utilização de diferentes dicionários léxicos. Não só isso, sugere-se também que pode ser estudada a relação de cointegração entre as variáveis e aplicar um modelo de Vetor de Correção de Erros (VECM) para entender o tempo que o sistema retorna para seu nível de equilíbrio, ou até mesmo, aplicação de técnicas de \textit{Machine Learning}, como, por exemplo, a modelagem de tópicos.